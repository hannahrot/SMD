\documentclass[11pt,a4paper]{article}

\usepackage{amsmath} %for mathemathic formulas
\usepackage{amssymb}
\usepackage[ngerman]{babel} %for the german language by the spellling reform (without the package the date would look like April 20, 2020)
\usepackage{enumitem} %for enumeration surrounding
\usepackage{graphicx} %for pictures
\usepackage{siunitx}

\title{Blatt 4}
\date{\today}
\author{Hannah Rotgeri \and Lena Olbrich}

\begin{document}
    \maketitle

    \section*{Aufgabe 1}
\subsection{Aufgabenteil a}
Zuerst Nummierung berechnen :
$1= \int_{-\pi}^{\pi} N e^{-\psi k} d\psi= N \left( [\frac{1}{k}e^{\psi k}]^0_{-\pi} + [-\frac{1}{k}e^{-\psi k}]^\pi_{0}\right)=\frac{2}{k}(1-e^{-\pi k}) \Leftrightarrow N=\frac{k}{2(1-e^{-\pi k})}
$

\subsection{Aufgabenteil b}
Kumulative Verteilung berechnen:
$A(x)=\int_{-\pi}^{\psi}e^{-|\psi|k}d\psi=
\int_{-\pi}^{0} e^{\psi k} d \psi + \int_{0}^{\psi} e^{-\psi k} d \psi
= [\frac{1}{k} e^{\psi k}]^0_{-\pi} +[-\frac{1}{k}e^{-\psi k}]^\psi_0
=\frac{1}{k}\left(2- e^{-\pi k} - e^{-\psi k} \right)
$\\
Normierte Fläche:
$r(x)= A(x)\cdot N = \left(2- e^{-\pi k} - e^{-\psi k} \right) \cdot \frac{1}{2(1-e^{-\pi k})}
$
\subsection{Aufgabenteil c}
Inverse berechnen:
$ \psi(r)=-\frac{1}{k}ln(2-2r(1-e^{-\pi k})-e^{-\pi k})
$
	\section*{Aufgabe 2}


\end{document}
